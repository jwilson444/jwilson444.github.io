% Options for packages loaded elsewhere
\PassOptionsToPackage{unicode}{hyperref}
\PassOptionsToPackage{hyphens}{url}
%
\documentclass[
]{article}
\usepackage{amsmath,amssymb}
\usepackage{iftex}
\ifPDFTeX
  \usepackage[T1]{fontenc}
  \usepackage[utf8]{inputenc}
  \usepackage{textcomp} % provide euro and other symbols
\else % if luatex or xetex
  \usepackage{unicode-math} % this also loads fontspec
  \defaultfontfeatures{Scale=MatchLowercase}
  \defaultfontfeatures[\rmfamily]{Ligatures=TeX,Scale=1}
\fi
\usepackage{lmodern}
\ifPDFTeX\else
  % xetex/luatex font selection
\fi
% Use upquote if available, for straight quotes in verbatim environments
\IfFileExists{upquote.sty}{\usepackage{upquote}}{}
\IfFileExists{microtype.sty}{% use microtype if available
  \usepackage[]{microtype}
  \UseMicrotypeSet[protrusion]{basicmath} % disable protrusion for tt fonts
}{}
\makeatletter
\@ifundefined{KOMAClassName}{% if non-KOMA class
  \IfFileExists{parskip.sty}{%
    \usepackage{parskip}
  }{% else
    \setlength{\parindent}{0pt}
    \setlength{\parskip}{6pt plus 2pt minus 1pt}}
}{% if KOMA class
  \KOMAoptions{parskip=half}}
\makeatother
\usepackage{xcolor}
\usepackage[margin=1in]{geometry}
\usepackage{color}
\usepackage{fancyvrb}
\newcommand{\VerbBar}{|}
\newcommand{\VERB}{\Verb[commandchars=\\\{\}]}
\DefineVerbatimEnvironment{Highlighting}{Verbatim}{commandchars=\\\{\}}
% Add ',fontsize=\small' for more characters per line
\usepackage{framed}
\definecolor{shadecolor}{RGB}{248,248,248}
\newenvironment{Shaded}{\begin{snugshade}}{\end{snugshade}}
\newcommand{\AlertTok}[1]{\textcolor[rgb]{0.94,0.16,0.16}{#1}}
\newcommand{\AnnotationTok}[1]{\textcolor[rgb]{0.56,0.35,0.01}{\textbf{\textit{#1}}}}
\newcommand{\AttributeTok}[1]{\textcolor[rgb]{0.13,0.29,0.53}{#1}}
\newcommand{\BaseNTok}[1]{\textcolor[rgb]{0.00,0.00,0.81}{#1}}
\newcommand{\BuiltInTok}[1]{#1}
\newcommand{\CharTok}[1]{\textcolor[rgb]{0.31,0.60,0.02}{#1}}
\newcommand{\CommentTok}[1]{\textcolor[rgb]{0.56,0.35,0.01}{\textit{#1}}}
\newcommand{\CommentVarTok}[1]{\textcolor[rgb]{0.56,0.35,0.01}{\textbf{\textit{#1}}}}
\newcommand{\ConstantTok}[1]{\textcolor[rgb]{0.56,0.35,0.01}{#1}}
\newcommand{\ControlFlowTok}[1]{\textcolor[rgb]{0.13,0.29,0.53}{\textbf{#1}}}
\newcommand{\DataTypeTok}[1]{\textcolor[rgb]{0.13,0.29,0.53}{#1}}
\newcommand{\DecValTok}[1]{\textcolor[rgb]{0.00,0.00,0.81}{#1}}
\newcommand{\DocumentationTok}[1]{\textcolor[rgb]{0.56,0.35,0.01}{\textbf{\textit{#1}}}}
\newcommand{\ErrorTok}[1]{\textcolor[rgb]{0.64,0.00,0.00}{\textbf{#1}}}
\newcommand{\ExtensionTok}[1]{#1}
\newcommand{\FloatTok}[1]{\textcolor[rgb]{0.00,0.00,0.81}{#1}}
\newcommand{\FunctionTok}[1]{\textcolor[rgb]{0.13,0.29,0.53}{\textbf{#1}}}
\newcommand{\ImportTok}[1]{#1}
\newcommand{\InformationTok}[1]{\textcolor[rgb]{0.56,0.35,0.01}{\textbf{\textit{#1}}}}
\newcommand{\KeywordTok}[1]{\textcolor[rgb]{0.13,0.29,0.53}{\textbf{#1}}}
\newcommand{\NormalTok}[1]{#1}
\newcommand{\OperatorTok}[1]{\textcolor[rgb]{0.81,0.36,0.00}{\textbf{#1}}}
\newcommand{\OtherTok}[1]{\textcolor[rgb]{0.56,0.35,0.01}{#1}}
\newcommand{\PreprocessorTok}[1]{\textcolor[rgb]{0.56,0.35,0.01}{\textit{#1}}}
\newcommand{\RegionMarkerTok}[1]{#1}
\newcommand{\SpecialCharTok}[1]{\textcolor[rgb]{0.81,0.36,0.00}{\textbf{#1}}}
\newcommand{\SpecialStringTok}[1]{\textcolor[rgb]{0.31,0.60,0.02}{#1}}
\newcommand{\StringTok}[1]{\textcolor[rgb]{0.31,0.60,0.02}{#1}}
\newcommand{\VariableTok}[1]{\textcolor[rgb]{0.00,0.00,0.00}{#1}}
\newcommand{\VerbatimStringTok}[1]{\textcolor[rgb]{0.31,0.60,0.02}{#1}}
\newcommand{\WarningTok}[1]{\textcolor[rgb]{0.56,0.35,0.01}{\textbf{\textit{#1}}}}
\usepackage{graphicx}
\makeatletter
\def\maxwidth{\ifdim\Gin@nat@width>\linewidth\linewidth\else\Gin@nat@width\fi}
\def\maxheight{\ifdim\Gin@nat@height>\textheight\textheight\else\Gin@nat@height\fi}
\makeatother
% Scale images if necessary, so that they will not overflow the page
% margins by default, and it is still possible to overwrite the defaults
% using explicit options in \includegraphics[width, height, ...]{}
\setkeys{Gin}{width=\maxwidth,height=\maxheight,keepaspectratio}
% Set default figure placement to htbp
\makeatletter
\def\fps@figure{htbp}
\makeatother
\setlength{\emergencystretch}{3em} % prevent overfull lines
\providecommand{\tightlist}{%
  \setlength{\itemsep}{0pt}\setlength{\parskip}{0pt}}
\setcounter{secnumdepth}{-\maxdimen} % remove section numbering
\ifLuaTeX
  \usepackage{selnolig}  % disable illegal ligatures
\fi
\usepackage{bookmark}
\IfFileExists{xurl.sty}{\usepackage{xurl}}{} % add URL line breaks if available
\urlstyle{same}
\hypersetup{
  pdftitle={An analysis of global ecological footprint},
  pdfauthor={Jeremy Wilson},
  hidelinks,
  pdfcreator={LaTeX via pandoc}}

\title{An analysis of global ecological footprint}
\author{Jeremy Wilson}
\date{2025-04-22}

\begin{document}
\maketitle

{
\setcounter{tocdepth}{2}
\tableofcontents
}
\begin{center}\rule{0.5\linewidth}{0.5pt}\end{center}

\subsection{Introduction}\label{introduction}

The goal of this report is to explore and analyze a dataset regarding
the ecological footprint of the world. If possible, specific
characteristics will be identified that correlate with ecological
impact. This page will first give some background information regarding
the dataset and relevant info for the reader. Then, it will explore a
series of analytical questions. Following the exploratory analysis, a
predictive model will be developed and tested. I hope you enjoy.

\subsection{Background and Data}\label{background-and-data}

The dataset countries.csv was obtained from Kaggle at the following
link:
\url{https://www.kaggle.com/datasets/footprintnetwork/ecological-footprint}
The data therein was obtained from the Global Footprint Network, an
international organization that tracks and works on issues of global
ecological importance. The dataset is rather extensive, and so a
glossary is provided below, simply click each term as needed to expand
it and see the definition.

\subsubsection{Glossary}\label{glossary}

Country

The nation-state or territory for which the ecological data is reported.

Region

A geographic area comprising multiple countries, often grouped for
analytical purposes.

Population in Millions

The total number of people residing in the country, expressed in
millions.

HDI

A composite index measuring average achievement in key dimensions of
human development: health, education, and standard of living.

GDP per Capita

The gross domestic product divided by the midyear population, indicating
the average economic output per person.

Cropland Footprint

The amount of cropland required to grow crops for food and fiber, feed
for livestock, oil crops, and rubber.

Grazing Footprint

The area of grassland used to raise livestock for meat, dairy, hide, and
wool products.

Forest Footprint

The forest area required to supply timber, pulp, and fuel wood consumed
by a country.

Carbon Footprint

The amount of forest land required to absorb CO₂ emissions from fossil
fuel combustion.

Fish Footprint

The area of marine and inland water ecosystems required to support the
annual catch of fish.

Total Ecological Footprint

The sum of all resource demands placed on nature by a population.

Cropland

Biologically productive land used for growing crops for food, fiber, and
feed.

Grazing Land

Land used for raising livestock through grazing.

Forest Land

Land covered by forests that provide timber and absorb carbon.

Fishing Water

Marine and inland water bodies that provide fish and aquatic resources.

Urban Land

Land covered by human infrastructure and built environments.

Total Biocapacity

The capacity of ecosystems to regenerate what people demand from them.

Biocapacity Deficit or Reserve

The difference between a region's biocapacity and its Ecological
Footprint.

Earths Required

The number of Earths needed if everyone lived like this population.

Countries Required

The number of countries with similar biocapacity needed to support the
consumption.

Data Quality

An assessment of the reliability and accuracy of the underlying data.

It's also important to note that the dataset was cleaned and prepared to
facilitate more efficient analysis.

\subsection{Exploratory Analysis}\label{exploratory-analysis}

I will break it down by topic as for how I explored the dataset.

\subsubsection{1. What variables seem to correlate with ecological
footprint?}\label{what-variables-seem-to-correlate-with-ecological-footprint}

The most obvious and simple question, is there a single variable or
multiple variables that on their own correlate with ecological
footprint? First, let's look at the most important variable, the total
ecological footprint.

\paragraph{Total Ecological Footprint}\label{total-ecological-footprint}

\begin{Shaded}
\begin{Highlighting}[]
\NormalTok{total\_ecological\_correlations }\OtherTok{\textless{}{-}}\NormalTok{ pol }\SpecialCharTok{\%\textgreater{}\%}
  \FunctionTok{select}\NormalTok{(}\FunctionTok{where}\NormalTok{(is.numeric)) }\SpecialCharTok{\%\textgreater{}\%}
  \FunctionTok{summarise}\NormalTok{(}\FunctionTok{across}\NormalTok{(}\SpecialCharTok{{-}}\NormalTok{total\_ecological\_footprint, }
                   \SpecialCharTok{\textasciitilde{}} \FunctionTok{cor}\NormalTok{(.x, total\_ecological\_footprint, }\AttributeTok{use =} \StringTok{"complete.obs"}\NormalTok{))) }\SpecialCharTok{\%\textgreater{}\%}
  \FunctionTok{pivot\_longer}\NormalTok{(}\FunctionTok{everything}\NormalTok{(), }\AttributeTok{names\_to =} \StringTok{"predictor"}\NormalTok{, }\AttributeTok{values\_to =} \StringTok{"correlation"}\NormalTok{) }\SpecialCharTok{\%\textgreater{}\%}
  \FunctionTok{arrange}\NormalTok{(}\FunctionTok{desc}\NormalTok{(}\FunctionTok{abs}\NormalTok{(correlation)))}

\FunctionTok{print}\NormalTok{(total\_ecological\_correlations)}
\end{Highlighting}
\end{Shaded}

\begin{verbatim}
## # A tibble: 17 x 2
##    predictor                      correlation
##    <chr>                                <dbl>
##  1 earths_required                     1.00  
##  2 carbon_footprint                    0.952 
##  3 gdp_per_capita                      0.799 
##  4 hdi                                 0.739 
##  5 cropland_footprint                  0.588 
##  6 forest_footprint                    0.400 
##  7 urban_land                          0.357 
##  8 cropland                            0.303 
##  9 countries_required                  0.283 
## 10 grazing_footprint                   0.275 
## 11 fish_footprint                      0.193 
## 12 fishing_water                       0.145 
## 13 biocapacity_deficit_or_reserve     -0.135 
## 14 grazing_land                        0.0937
## 15 total_biocapacity                   0.0666
## 16 population_millions                -0.0577
## 17 forest_land                         0.0260
\end{verbatim}

We can ignore earths required as it is also a dependent variable like
total ecological footprint, and we are looking for possible causative or
explanatory variables that effect the total ecological footprint. With
that, it looks like carbon footprint has a considerable correlation,
next would be GDP per Capita and HDI.

Next, let's explore each footprint measured individually to see if one
variable specifically explains the others. \#\#\#\# Cropland Footprint

\begin{Shaded}
\begin{Highlighting}[]
\NormalTok{cropland\_correlations }\OtherTok{\textless{}{-}}\NormalTok{ pol }\SpecialCharTok{\%\textgreater{}\%}
  \FunctionTok{select}\NormalTok{(}\FunctionTok{where}\NormalTok{(is.numeric)) }\SpecialCharTok{\%\textgreater{}\%}
  \FunctionTok{summarise}\NormalTok{(}\FunctionTok{across}\NormalTok{(}\SpecialCharTok{{-}}\NormalTok{cropland\_footprint, }
                   \SpecialCharTok{\textasciitilde{}} \FunctionTok{cor}\NormalTok{(.x, cropland\_footprint, }\AttributeTok{use =} \StringTok{"complete.obs"}\NormalTok{))) }\SpecialCharTok{\%\textgreater{}\%}
  \FunctionTok{pivot\_longer}\NormalTok{(}\FunctionTok{everything}\NormalTok{(), }\AttributeTok{names\_to =} \StringTok{"predictor"}\NormalTok{, }\AttributeTok{values\_to =} \StringTok{"correlation"}\NormalTok{) }\SpecialCharTok{\%\textgreater{}\%}
  \FunctionTok{arrange}\NormalTok{(}\FunctionTok{desc}\NormalTok{(}\FunctionTok{abs}\NormalTok{(correlation)))}

\FunctionTok{print}\NormalTok{(cropland\_correlations)}
\end{Highlighting}
\end{Shaded}

\begin{verbatim}
## # A tibble: 17 x 2
##    predictor                      correlation
##    <chr>                                <dbl>
##  1 cropland                            0.765 
##  2 total_ecological_footprint          0.588 
##  3 earths_required                     0.588 
##  4 hdi                                 0.567 
##  5 gdp_per_capita                      0.506 
##  6 forest_footprint                    0.438 
##  7 urban_land                          0.432 
##  8 carbon_footprint                    0.410 
##  9 biocapacity_deficit_or_reserve     -0.0952
## 10 fish_footprint                      0.0802
## 11 grazing_footprint                   0.0779
## 12 fishing_water                       0.0770
## 13 grazing_land                        0.0747
## 14 forest_land                        -0.0515
## 15 countries_required                  0.0232
## 16 population_millions                -0.0196
## 17 total_biocapacity                   0.0162
\end{verbatim}

Cropland footprint seems to be most correlated with cropland, which
might seem obvious, but it is worth validating.

\paragraph{Grazing Footprint}\label{grazing-footprint}

\begin{Shaded}
\begin{Highlighting}[]
\NormalTok{grazing\_correlations }\OtherTok{\textless{}{-}}\NormalTok{ pol }\SpecialCharTok{\%\textgreater{}\%}
  \FunctionTok{select}\NormalTok{(}\FunctionTok{where}\NormalTok{(is.numeric)) }\SpecialCharTok{\%\textgreater{}\%}
  \FunctionTok{summarise}\NormalTok{(}\FunctionTok{across}\NormalTok{(}\SpecialCharTok{{-}}\NormalTok{grazing\_footprint, }
                   \SpecialCharTok{\textasciitilde{}} \FunctionTok{cor}\NormalTok{(.x, grazing\_footprint, }\AttributeTok{use =} \StringTok{"complete.obs"}\NormalTok{))) }\SpecialCharTok{\%\textgreater{}\%}
  \FunctionTok{pivot\_longer}\NormalTok{(}\FunctionTok{everything}\NormalTok{(), }\AttributeTok{names\_to =} \StringTok{"predictor"}\NormalTok{, }\AttributeTok{values\_to =} \StringTok{"correlation"}\NormalTok{) }\SpecialCharTok{\%\textgreater{}\%}
  \FunctionTok{arrange}\NormalTok{(}\FunctionTok{desc}\NormalTok{(}\FunctionTok{abs}\NormalTok{(correlation)))}

\FunctionTok{print}\NormalTok{(grazing\_correlations)}
\end{Highlighting}
\end{Shaded}

\begin{verbatim}
## # A tibble: 17 x 2
##    predictor                      correlation
##    <chr>                                <dbl>
##  1 grazing_land                        0.696 
##  2 total_ecological_footprint          0.275 
##  3 earths_required                     0.275 
##  4 carbon_footprint                    0.126 
##  5 gdp_per_capita                      0.105 
##  6 cropland                            0.104 
##  7 hdi                                 0.0922
##  8 total_biocapacity                   0.0897
##  9 urban_land                          0.0872
## 10 population_millions                -0.0849
## 11 cropland_footprint                  0.0779
## 12 fishing_water                       0.0741
## 13 forest_footprint                    0.0570
## 14 fish_footprint                     -0.0555
## 15 biocapacity_deficit_or_reserve      0.0372
## 16 countries_required                 -0.0299
## 17 forest_land                         0.0174
\end{verbatim}

Once again, grazing land correlates with grazing footprint the best.

\paragraph{Forest Footprint}\label{forest-footprint}

\begin{Shaded}
\begin{Highlighting}[]
\NormalTok{forest\_correlations }\OtherTok{\textless{}{-}}\NormalTok{ pol }\SpecialCharTok{\%\textgreater{}\%}
  \FunctionTok{select}\NormalTok{(}\FunctionTok{where}\NormalTok{(is.numeric)) }\SpecialCharTok{\%\textgreater{}\%}
  \FunctionTok{summarise}\NormalTok{(}\FunctionTok{across}\NormalTok{(}\SpecialCharTok{{-}}\NormalTok{forest\_footprint, }
                   \SpecialCharTok{\textasciitilde{}} \FunctionTok{cor}\NormalTok{(.x, forest\_footprint, }\AttributeTok{use =} \StringTok{"complete.obs"}\NormalTok{))) }\SpecialCharTok{\%\textgreater{}\%}
  \FunctionTok{pivot\_longer}\NormalTok{(}\FunctionTok{everything}\NormalTok{(), }\AttributeTok{names\_to =} \StringTok{"predictor"}\NormalTok{, }\AttributeTok{values\_to =} \StringTok{"correlation"}\NormalTok{) }\SpecialCharTok{\%\textgreater{}\%}
  \FunctionTok{arrange}\NormalTok{(}\FunctionTok{desc}\NormalTok{(}\FunctionTok{abs}\NormalTok{(correlation)))}

\FunctionTok{print}\NormalTok{(forest\_correlations)}
\end{Highlighting}
\end{Shaded}

\begin{verbatim}
## # A tibble: 17 x 2
##    predictor                      correlation
##    <chr>                                <dbl>
##  1 urban_land                          0.457 
##  2 cropland_footprint                  0.438 
##  3 cropland                            0.416 
##  4 earths_required                     0.400 
##  5 total_ecological_footprint          0.400 
##  6 gdp_per_capita                      0.278 
##  7 hdi                                 0.231 
##  8 carbon_footprint                    0.195 
##  9 total_biocapacity                   0.173 
## 10 forest_land                         0.143 
## 11 fishing_water                       0.112 
## 12 biocapacity_deficit_or_reserve      0.0960
## 13 grazing_land                        0.0867
## 14 countries_required                  0.0805
## 15 population_millions                -0.0779
## 16 grazing_footprint                   0.0570
## 17 fish_footprint                     -0.0420
\end{verbatim}

This might seem different at first glance, but it is also obvious. The
more urban land and cropland developed by individuals/societies, the
more forest must be cleared away, the greater the footprint on forests.

\paragraph{Carbon Footprint}\label{carbon-footprint}

\begin{Shaded}
\begin{Highlighting}[]
\NormalTok{carbon\_correlations }\OtherTok{\textless{}{-}}\NormalTok{ pol }\SpecialCharTok{\%\textgreater{}\%}
  \FunctionTok{select}\NormalTok{(}\FunctionTok{where}\NormalTok{(is.numeric)) }\SpecialCharTok{\%\textgreater{}\%}
  \FunctionTok{summarise}\NormalTok{(}\FunctionTok{across}\NormalTok{(}\SpecialCharTok{{-}}\NormalTok{carbon\_footprint, }
                   \SpecialCharTok{\textasciitilde{}} \FunctionTok{cor}\NormalTok{(.x, carbon\_footprint, }\AttributeTok{use =} \StringTok{"complete.obs"}\NormalTok{))) }\SpecialCharTok{\%\textgreater{}\%}
  \FunctionTok{pivot\_longer}\NormalTok{(}\FunctionTok{everything}\NormalTok{(), }\AttributeTok{names\_to =} \StringTok{"predictor"}\NormalTok{, }\AttributeTok{values\_to =} \StringTok{"correlation"}\NormalTok{) }\SpecialCharTok{\%\textgreater{}\%}
  \FunctionTok{arrange}\NormalTok{(}\FunctionTok{desc}\NormalTok{(}\FunctionTok{abs}\NormalTok{(correlation)))}

\FunctionTok{print}\NormalTok{(carbon\_correlations)}
\end{Highlighting}
\end{Shaded}

\begin{verbatim}
## # A tibble: 17 x 2
##    predictor                      correlation
##    <chr>                                <dbl>
##  1 earths_required                     0.952 
##  2 total_ecological_footprint          0.952 
##  3 gdp_per_capita                      0.824 
##  4 hdi                                 0.699 
##  5 cropland_footprint                  0.410 
##  6 countries_required                  0.307 
##  7 urban_land                          0.224 
##  8 forest_footprint                    0.195 
##  9 fish_footprint                      0.156 
## 10 biocapacity_deficit_or_reserve     -0.152 
## 11 grazing_footprint                   0.126 
## 12 cropland                            0.119 
## 13 fishing_water                       0.113 
## 14 grazing_land                       -0.0418
## 15 total_biocapacity                   0.0288
## 16 population_millions                -0.0212
## 17 forest_land                         0.0104
\end{verbatim}

This variable is the first individual footprint to correlate strongly
with GDP per Capita.

\subsubsection{2. Ecological Sustainability
Gaps}\label{ecological-sustainability-gaps}

The goal of this section is to identify what countries have the largest
biocapacity deficits and reserves. This is best visualized by viewing
the countries in deficit separate from those with a reserve.

\paragraph{Countries with a Biocapacity
Deficiency}\label{countries-with-a-biocapacity-deficiency}

\begin{Shaded}
\begin{Highlighting}[]
\NormalTok{pol2\_country\_deficit }\OtherTok{\textless{}{-}}\NormalTok{ pol }\SpecialCharTok{\%\textgreater{}\%}
  \FunctionTok{filter}\NormalTok{(biocapacity\_deficit\_or\_reserve }\SpecialCharTok{\textless{}} \DecValTok{0}\NormalTok{) }\SpecialCharTok{\%\textgreater{}\%}
  \FunctionTok{arrange}\NormalTok{(biocapacity\_deficit\_or\_reserve) }\SpecialCharTok{\%\textgreater{}\%}
  \FunctionTok{mutate}\NormalTok{(}\AttributeTok{country =} \FunctionTok{factor}\NormalTok{(country, }\AttributeTok{levels =}\NormalTok{ country),}
         \AttributeTok{half\_group =} \FunctionTok{if\_else}\NormalTok{(}\FunctionTok{row\_number}\NormalTok{() }\SpecialCharTok{\textless{}=} \FunctionTok{n}\NormalTok{() }\SpecialCharTok{/} \DecValTok{2}\NormalTok{, }\StringTok{"Most Severe"}\NormalTok{, }\StringTok{"Less Severe"}\NormalTok{)) }\SpecialCharTok{\%\textgreater{}\%}
  \FunctionTok{ggplot}\NormalTok{(}\FunctionTok{aes}\NormalTok{(}\AttributeTok{x =}\NormalTok{ country, }\AttributeTok{y =}\NormalTok{ biocapacity\_deficit\_or\_reserve)) }\SpecialCharTok{+}
  \FunctionTok{geom\_col}\NormalTok{(}\AttributeTok{fill =} \StringTok{"firebrick"}\NormalTok{) }\SpecialCharTok{+}
  \FunctionTok{coord\_flip}\NormalTok{() }\SpecialCharTok{+}
  \FunctionTok{scale\_y\_reverse}\NormalTok{() }\SpecialCharTok{+}
  \FunctionTok{facet\_wrap}\NormalTok{(}\SpecialCharTok{\textasciitilde{}}\NormalTok{half\_group, }\AttributeTok{scales =} \StringTok{"free\_y"}\NormalTok{) }\SpecialCharTok{+}
  \FunctionTok{labs}\NormalTok{(}\AttributeTok{title =} \StringTok{"Countries in Biocapacity Deficit"}\NormalTok{,}
       \AttributeTok{subtitle =} \StringTok{"Split by Severity: Most Severe vs Less Severe"}\NormalTok{,}
       \AttributeTok{y =} \StringTok{"Biocapacity Deficit"}\NormalTok{,}
       \AttributeTok{x =} \ConstantTok{NULL}\NormalTok{) }\SpecialCharTok{+}
  \FunctionTok{theme\_minimal}\NormalTok{() }\SpecialCharTok{+}
  \FunctionTok{theme}\NormalTok{(}\AttributeTok{axis.text.x =} \FunctionTok{element\_text}\NormalTok{(}\AttributeTok{size =} \DecValTok{6}\NormalTok{, }\AttributeTok{hjust =} \DecValTok{1}\NormalTok{),}
        \AttributeTok{strip.text =} \FunctionTok{element\_text}\NormalTok{(}\AttributeTok{face =} \StringTok{"bold"}\NormalTok{, }\AttributeTok{size =} \DecValTok{10}\NormalTok{))}
\NormalTok{pol2\_country\_deficit}
\end{Highlighting}
\end{Shaded}

\includegraphics{Final_Project_files/figure-latex/pol2_country_deficit-1.pdf}

\paragraph{Countries with a Biocapacity
Reserve}\label{countries-with-a-biocapacity-reserve}

\begin{Shaded}
\begin{Highlighting}[]
\NormalTok{pol2\_country\_reserve }\OtherTok{\textless{}{-}}\NormalTok{ pol }\SpecialCharTok{\%\textgreater{}\%}
  \FunctionTok{filter}\NormalTok{(biocapacity\_deficit\_or\_reserve }\SpecialCharTok{\textgreater{}} \DecValTok{0}\NormalTok{) }\SpecialCharTok{\%\textgreater{}\%}
  \FunctionTok{arrange}\NormalTok{(biocapacity\_deficit\_or\_reserve) }\SpecialCharTok{\%\textgreater{}\%}
  \FunctionTok{mutate}\NormalTok{(}\AttributeTok{country =} \FunctionTok{factor}\NormalTok{(country, }\AttributeTok{levels =}\NormalTok{ country)) }\SpecialCharTok{\%\textgreater{}\%}
  \FunctionTok{ggplot}\NormalTok{(}\FunctionTok{aes}\NormalTok{(}\AttributeTok{x =}\NormalTok{ country, }\AttributeTok{y =}\NormalTok{ biocapacity\_deficit\_or\_reserve)) }\SpecialCharTok{+}
  \FunctionTok{geom\_col}\NormalTok{(}\AttributeTok{fill =} \StringTok{"forestgreen"}\NormalTok{) }\SpecialCharTok{+}
  \FunctionTok{labs}\NormalTok{(}\AttributeTok{title =} \StringTok{"Countries in Biocapacity Reserve"}\NormalTok{,}
       \AttributeTok{y =} \StringTok{"Biocapacity Reserve"}\NormalTok{,}
       \AttributeTok{x =} \ConstantTok{NULL}\NormalTok{) }\SpecialCharTok{+}
  \FunctionTok{theme\_minimal}\NormalTok{() }\SpecialCharTok{+}
  \FunctionTok{theme}\NormalTok{(}\AttributeTok{axis.text.x =} \FunctionTok{element\_text}\NormalTok{(}\AttributeTok{angle =} \DecValTok{70}\NormalTok{, }\AttributeTok{size =} \DecValTok{6}\NormalTok{, }\AttributeTok{hjust =} \DecValTok{1}\NormalTok{))}
\NormalTok{pol2\_country\_reserve}
\end{Highlighting}
\end{Shaded}

\includegraphics{Final_Project_files/figure-latex/pol2_country_reserve-1.pdf}

These two plots demonstrate what countries have the greatest deficits
and reserves. It is apparent that few countries, 27.81\%, have a
biocapacity reserve, with the other 72.19\% having a deficit.

We can also observe the same factor by region.

\paragraph{Biocapacity Reserve or Deficit by
Region}\label{biocapacity-reserve-or-deficit-by-region}

\begin{Shaded}
\begin{Highlighting}[]
\NormalTok{pol2region\_bar }\OtherTok{\textless{}{-}}\NormalTok{ pol }\SpecialCharTok{\%\textgreater{}\%}
  \FunctionTok{group\_by}\NormalTok{(region) }\SpecialCharTok{\%\textgreater{}\%}
  \FunctionTok{summarise}\NormalTok{(}\AttributeTok{total\_deficit\_or\_reserve =} \FunctionTok{sum}\NormalTok{(biocapacity\_deficit\_or\_reserve, }\AttributeTok{na.rm =} \ConstantTok{TRUE}\NormalTok{)) }\SpecialCharTok{\%\textgreater{}\%}
  \FunctionTok{arrange}\NormalTok{(total\_deficit\_or\_reserve) }\SpecialCharTok{\%\textgreater{}\%}
  \FunctionTok{mutate}\NormalTok{(}\AttributeTok{region =} \FunctionTok{factor}\NormalTok{(region, }\AttributeTok{levels =}\NormalTok{ region)) }\SpecialCharTok{\%\textgreater{}\%}
  \FunctionTok{ggplot}\NormalTok{(}\FunctionTok{aes}\NormalTok{(}\AttributeTok{x =}\NormalTok{ region, }\AttributeTok{y =}\NormalTok{ total\_deficit\_or\_reserve, }\AttributeTok{fill =}\NormalTok{ total\_deficit\_or\_reserve }\SpecialCharTok{\textgreater{}} \DecValTok{0}\NormalTok{)) }\SpecialCharTok{+}
  \FunctionTok{geom\_col}\NormalTok{() }\SpecialCharTok{+}
  \FunctionTok{scale\_fill\_manual}\NormalTok{(}\AttributeTok{values =} \FunctionTok{c}\NormalTok{(}\StringTok{"TRUE"} \OtherTok{=} \StringTok{"forestgreen"}\NormalTok{, }\StringTok{"FALSE"} \OtherTok{=} \StringTok{"darkred"}\NormalTok{),}
                    \AttributeTok{labels =} \FunctionTok{c}\NormalTok{(}\StringTok{"Deficit"}\NormalTok{, }\StringTok{"Reserve"}\NormalTok{)) }\SpecialCharTok{+}
  \FunctionTok{labs}\NormalTok{(}\AttributeTok{title =} \StringTok{"Total Biocapacity Deficit or Reserve by Region"}\NormalTok{,}
       \AttributeTok{y =} \StringTok{"Total Biocapacity"}\NormalTok{,}
       \AttributeTok{x =} \StringTok{"Region"}\NormalTok{,}
       \AttributeTok{fill =} \StringTok{""}\NormalTok{) }\SpecialCharTok{+}
  \FunctionTok{theme\_minimal}\NormalTok{() }\SpecialCharTok{+}
  \FunctionTok{theme}\NormalTok{(}\AttributeTok{axis.text.x =} \FunctionTok{element\_text}\NormalTok{(}\AttributeTok{color =} \StringTok{"black"}\NormalTok{, }\AttributeTok{angle =} \DecValTok{50}\NormalTok{, }\AttributeTok{size =} \DecValTok{8}\NormalTok{, }\AttributeTok{hjust =} \DecValTok{1}\NormalTok{)) }\SpecialCharTok{+}
  \FunctionTok{coord\_cartesian}\NormalTok{(}\AttributeTok{ylim =} \FunctionTok{c}\NormalTok{(}\SpecialCharTok{{-}}\DecValTok{100}\NormalTok{, }\DecValTok{250}\NormalTok{))}
\NormalTok{pol2region\_bar}
\end{Highlighting}
\end{Shaded}

\includegraphics{Final_Project_files/figure-latex/pol2region_bar-1.pdf}

It is apparent that the top eco-footprint offenders are those in Asia
and Europe, while Latin America appears to have a monopoly on
Biocapacity reserve.

\subsubsection{3. Human Development vs.~Environmental
Impact}\label{human-development-vs.-environmental-impact}

This section will explore the question: Is there a tradeoff between
quality of life, GDP per capita, and sustainability?

\begin{Shaded}
\begin{Highlighting}[]
\NormalTok{pol3 }\OtherTok{\textless{}{-}}\NormalTok{ pol }\SpecialCharTok{\%\textgreater{}\%}
  \FunctionTok{ggplot}\NormalTok{(}\FunctionTok{aes}\NormalTok{(}\AttributeTok{x =}\NormalTok{ hdi, }\AttributeTok{y =}\NormalTok{ total\_ecological\_footprint)) }\SpecialCharTok{+}
  \FunctionTok{geom\_point}\NormalTok{(}\AttributeTok{color =} \StringTok{"black"}\NormalTok{, }\AttributeTok{alpha =} \FloatTok{0.6}\NormalTok{) }\SpecialCharTok{+}
  \FunctionTok{geom\_smooth}\NormalTok{(}\AttributeTok{color =} \StringTok{"green4"}\NormalTok{) }\SpecialCharTok{+}
  \FunctionTok{theme\_minimal}\NormalTok{() }\SpecialCharTok{+}
  \FunctionTok{labs}\NormalTok{( }\AttributeTok{title =} \StringTok{"Human Development Index vs. Total Ecological Footprint"}\NormalTok{,}
        \AttributeTok{x =} \StringTok{"Human Development Index"}\NormalTok{,}
        \AttributeTok{y =} \StringTok{"Total Ecological Footprint"}\NormalTok{)}
\NormalTok{pol3}
\end{Highlighting}
\end{Shaded}

\begin{verbatim}
## `geom_smooth()` using method = 'loess' and formula = 'y ~ x'
\end{verbatim}

\begin{verbatim}
## Warning: Removed 16 rows containing non-finite outside the scale range
## (`stat_smooth()`).
\end{verbatim}

\begin{verbatim}
## Warning: Removed 16 rows containing missing values or values outside the scale range
## (`geom_point()`).
\end{verbatim}

\includegraphics{Final_Project_files/figure-latex/unnamed-chunk-6-1.pdf}

The plot shows an upward trend that as HDI increases, total ecological
footprint also increases.

\begin{Shaded}
\begin{Highlighting}[]
\NormalTok{pol4 }\OtherTok{\textless{}{-}}\NormalTok{ pol }\SpecialCharTok{\%\textgreater{}\%}
  \FunctionTok{ggplot}\NormalTok{(}\FunctionTok{aes}\NormalTok{(}\AttributeTok{x =}\NormalTok{ gdp\_per\_capita, }\AttributeTok{y =}\NormalTok{ total\_ecological\_footprint)) }\SpecialCharTok{+}
  \FunctionTok{geom\_point}\NormalTok{() }\SpecialCharTok{+}
  \FunctionTok{geom\_smooth}\NormalTok{() }\SpecialCharTok{+}
  \FunctionTok{theme\_minimal}\NormalTok{()}
\end{Highlighting}
\end{Shaded}

\subsubsection{4. Earths Required
vs.~Footprint}\label{earths-required-vs.-footprint}

Another way to view the data is by viewing earths required by footprint.
This could give insight as to what footprint is associated with the
worst consumption/pollution styles.

\begin{Shaded}
\begin{Highlighting}[]
\NormalTok{Earths\_required\_vs\_footprints }\OtherTok{\textless{}{-}}\NormalTok{ pol }\SpecialCharTok{\%\textgreater{}\%}
  \FunctionTok{select}\NormalTok{(earths\_required, }\FunctionTok{ends\_with}\NormalTok{(}\StringTok{\textquotesingle{}\_footprint\textquotesingle{}}\NormalTok{)) }\SpecialCharTok{\%\textgreater{}\%}
  \FunctionTok{summarise}\NormalTok{(}\FunctionTok{across}\NormalTok{(}\SpecialCharTok{{-}}\NormalTok{earths\_required, }
                   \SpecialCharTok{\textasciitilde{}} \FunctionTok{cor}\NormalTok{(.x, earths\_required, }\AttributeTok{use =} \StringTok{"complete.obs"}\NormalTok{))) }\SpecialCharTok{\%\textgreater{}\%}
  \FunctionTok{pivot\_longer}\NormalTok{(}\FunctionTok{everything}\NormalTok{(), }\AttributeTok{names\_to =} \StringTok{"predictor"}\NormalTok{, }\AttributeTok{values\_to =} \StringTok{"correlation"}\NormalTok{) }\SpecialCharTok{\%\textgreater{}\%}
  \FunctionTok{arrange}\NormalTok{(}\FunctionTok{desc}\NormalTok{(}\FunctionTok{abs}\NormalTok{(correlation)))}

\FunctionTok{print}\NormalTok{(Earths\_required\_vs\_footprints)}
\end{Highlighting}
\end{Shaded}

\begin{verbatim}
## # A tibble: 6 x 2
##   predictor                  correlation
##   <chr>                            <dbl>
## 1 total_ecological_footprint       1.00 
## 2 carbon_footprint                 0.952
## 3 cropland_footprint               0.588
## 4 forest_footprint                 0.400
## 5 grazing_footprint                0.275
## 6 fish_footprint                   0.193
\end{verbatim}

It looks like carbon footprint is the most significant predictor of poor
ecological impact.

\subsection{Building a Predictive
Model}\label{building-a-predictive-model}

The prior analyses only compared single variables, but we know in real
life, many factors contribute to an outcome. This is an important topic,
as understanding these relationships is essential in order to build
meaningful models and make predictions of the dataset. In this section,
we will explore what predictive model best fits the data. More
specifically, we will try to predict the total ecological footprint
based on the variables that explain it best.

\subsubsection{Building various models}\label{building-various-models}

Based on the very first correlation analysis above, it's a good idea to
start with the most correlated depended variables and expand from there.
You will see below in my code that there are four different models.
These four models were built in sequence after testing many other
variable relationships.

\begin{Shaded}
\begin{Highlighting}[]
\FunctionTok{library}\NormalTok{(MASS)}
\end{Highlighting}
\end{Shaded}

\begin{verbatim}
## 
## Attaching package: 'MASS'
\end{verbatim}

\begin{verbatim}
## The following object is masked from 'package:patchwork':
## 
##     area
\end{verbatim}

\begin{verbatim}
## The following object is masked from 'package:dplyr':
## 
##     select
\end{verbatim}

\begin{Shaded}
\begin{Highlighting}[]
\NormalTok{mod1 }\OtherTok{\textless{}{-}} \FunctionTok{glm}\NormalTok{(}\AttributeTok{data =}\NormalTok{ pol,}
            \AttributeTok{formula =}\NormalTok{ total\_ecological\_footprint }\SpecialCharTok{\textasciitilde{}}\NormalTok{ population\_millions }\SpecialCharTok{*}\NormalTok{ gdp\_per\_capita }\SpecialCharTok{*}\NormalTok{ cropland\_footprint,}
            \AttributeTok{family =} \StringTok{\textquotesingle{}gaussian\textquotesingle{}}\NormalTok{)}

\NormalTok{mod2 }\OtherTok{\textless{}{-}} \FunctionTok{glm}\NormalTok{(}\AttributeTok{data =}\NormalTok{ pol,}
            \AttributeTok{formula =}\NormalTok{ total\_ecological\_footprint }\SpecialCharTok{\textasciitilde{}}\NormalTok{ population\_millions }\SpecialCharTok{*}\NormalTok{ hdi }\SpecialCharTok{*}\NormalTok{ gdp\_per\_capita }\SpecialCharTok{*}\NormalTok{ urban\_land,}
            \AttributeTok{family =} \StringTok{\textquotesingle{}gaussian\textquotesingle{}}\NormalTok{)}

\NormalTok{mod3 }\OtherTok{\textless{}{-}} \FunctionTok{glm}\NormalTok{(}\AttributeTok{data =}\NormalTok{ pol,}
            \AttributeTok{formula =}\NormalTok{ total\_ecological\_footprint }\SpecialCharTok{\textasciitilde{}}\NormalTok{ population\_millions }\SpecialCharTok{*}\NormalTok{ hdi }\SpecialCharTok{*}\NormalTok{ gdp\_per\_capita }\SpecialCharTok{*}\NormalTok{ forest\_land,}
            \AttributeTok{family =} \StringTok{\textquotesingle{}gaussian\textquotesingle{}}\NormalTok{)}

\NormalTok{mod4 }\OtherTok{\textless{}{-}} \FunctionTok{glm}\NormalTok{(}\AttributeTok{data =}\NormalTok{ pol,}
            \AttributeTok{formula =}\NormalTok{ total\_ecological\_footprint }\SpecialCharTok{\textasciitilde{}}\NormalTok{ population\_millions }\SpecialCharTok{*}\NormalTok{ hdi }\SpecialCharTok{*}\NormalTok{ gdp\_per\_capita }\SpecialCharTok{*}\NormalTok{ cropland\_footprint }\SpecialCharTok{*}\NormalTok{ urban\_land }\SpecialCharTok{*}\NormalTok{ forest\_land,}
            \AttributeTok{family =} \StringTok{\textquotesingle{}gaussian\textquotesingle{}}\NormalTok{)}
\end{Highlighting}
\end{Shaded}

\subsubsection{Comparing the Moodels}\label{comparing-the-moodels}

\begin{Shaded}
\begin{Highlighting}[]
\FunctionTok{compare\_performance}\NormalTok{(mod1, mod2, mod3, mod4)}
\end{Highlighting}
\end{Shaded}

\begin{verbatim}
## When comparing models, please note that probably not all models were fit
##   from same data.
\end{verbatim}

\begin{verbatim}
## # Comparison of Model Performance Indices
## 
## Name | Model | AIC (weights) | AICc (weights) | BIC (weights) |    R2 |  RMSE | Sigma
## -------------------------------------------------------------------------------------
## mod1 |   glm | 528.4 (<.001) |  529.6 (<.001) | 556.3 (0.079) | 0.755 | 1.158 | 1.188
## mod2 |   glm | 501.0 (<.001) |  505.3 (0.341) | 553.5 (0.314) | 0.810 | 1.023 | 1.077
## mod3 |   glm | 499.7 (<.001) |  504.0 (0.659) | 552.2 (0.606) | 0.811 | 1.019 | 1.073
## mod4 |   glm | 474.7 (>.999) |  564.1 (<.001) | 675.4 (<.001) | 0.911 | 0.701 | 0.901
\end{verbatim}

\begin{Shaded}
\begin{Highlighting}[]
\FunctionTok{compare\_performance}\NormalTok{(mod1, mod2, mod3, mod4) }\SpecialCharTok{\%\textgreater{}\%} \FunctionTok{plot}\NormalTok{()}
\end{Highlighting}
\end{Shaded}

\begin{verbatim}
## When comparing models, please note that probably not all models were fit
##   from same data.
\end{verbatim}

\includegraphics{Final_Project_files/figure-latex/unnamed-chunk-10-1.pdf}

The first table shows important statistical values for each model, and
the best model should have a low AIC, AICc, Sigma, and an R2 value
closest to 1. The radar chart helps show in a visual manner which model
has the best values for each value. The greater the diameter, or
proximity of each point to their respective vales, the more ideal the
value for that model. So looking at the table and the chart, it is clear
that model 4 is the best model for the dataset.

\subsubsection{Measuring Predictive
Accuracy}\label{measuring-predictive-accuracy}

\begin{Shaded}
\begin{Highlighting}[]
\NormalTok{cv }\OtherTok{\textless{}{-}} \FunctionTok{createDataPartition}\NormalTok{(pol}\SpecialCharTok{$}\NormalTok{total\_ecological\_footprint, }\AttributeTok{p =} \FloatTok{0.8}\NormalTok{, }\AttributeTok{list =}\NormalTok{ F)}
\NormalTok{pol\_train }\OtherTok{\textless{}{-}}\NormalTok{ pol[cv, ]}
\NormalTok{pol\_test }\OtherTok{\textless{}{-}}\NormalTok{ pol[}\SpecialCharTok{{-}}\NormalTok{cv,]}
\FunctionTok{dim}\NormalTok{(pol\_train)}
\end{Highlighting}
\end{Shaded}

\begin{verbatim}
## [1] 152  21
\end{verbatim}

\begin{Shaded}
\begin{Highlighting}[]
\FunctionTok{dim}\NormalTok{(pol) }
\end{Highlighting}
\end{Shaded}

\begin{verbatim}
## [1] 188  21
\end{verbatim}

\begin{Shaded}
\begin{Highlighting}[]
\NormalTok{train\_mod4 }\OtherTok{\textless{}{-}} \FunctionTok{glm}\NormalTok{(}\AttributeTok{data =}\NormalTok{ pol\_train,}
                 \AttributeTok{formula =}\NormalTok{ mod4}\SpecialCharTok{$}\NormalTok{formula,}
                 \AttributeTok{family =} \StringTok{\textquotesingle{}gaussian\textquotesingle{}}\NormalTok{)}

\NormalTok{pol\_test}\SpecialCharTok{$}\NormalTok{pred4 }\OtherTok{\textless{}{-}} \FunctionTok{predict}\NormalTok{(train\_mod4, pol\_test, }\AttributeTok{type =} \StringTok{\textquotesingle{}response\textquotesingle{}}\NormalTok{)}

\NormalTok{pol\_test\_clean }\OtherTok{\textless{}{-}}\NormalTok{ pol\_test }\SpecialCharTok{\%\textgreater{}\%} 
  \FunctionTok{filter}\NormalTok{(}\SpecialCharTok{!}\FunctionTok{is.na}\NormalTok{(population\_millions)) }\SpecialCharTok{\%\textgreater{}\%}
  \FunctionTok{filter}\NormalTok{(}\SpecialCharTok{!}\FunctionTok{is.na}\NormalTok{(hdi)) }\SpecialCharTok{\%\textgreater{}\%}
  \FunctionTok{filter}\NormalTok{(}\SpecialCharTok{!}\FunctionTok{is.na}\NormalTok{(gdp\_per\_capita)) }\SpecialCharTok{\%\textgreater{}\%}
  \FunctionTok{filter}\NormalTok{(}\SpecialCharTok{!}\FunctionTok{is.na}\NormalTok{(cropland\_footprint)) }\SpecialCharTok{\%\textgreater{}\%}
  \FunctionTok{filter}\NormalTok{(}\SpecialCharTok{!}\FunctionTok{is.na}\NormalTok{(urban\_land)) }\SpecialCharTok{\%\textgreater{}\%}
  \FunctionTok{filter}\NormalTok{(}\SpecialCharTok{!}\FunctionTok{is.na}\NormalTok{(forest\_land))}

\NormalTok{pol\_test\_clean}\SpecialCharTok{$}\NormalTok{pred4 }\OtherTok{\textless{}{-}} \FunctionTok{predict}\NormalTok{(train\_mod4, }\AttributeTok{newdata =}\NormalTok{ pol\_test\_clean, }\AttributeTok{type =} \StringTok{"response"}\NormalTok{)}

\FunctionTok{ggplot}\NormalTok{(pol\_test, }\FunctionTok{aes}\NormalTok{(}\AttributeTok{x =}\NormalTok{ total\_ecological\_footprint, }\AttributeTok{y =}\NormalTok{ pred4)) }\SpecialCharTok{+}
  \FunctionTok{geom\_point}\NormalTok{(}\AttributeTok{color =} \StringTok{"darkblue"}\NormalTok{) }\SpecialCharTok{+}
  \FunctionTok{geom\_abline}\NormalTok{(}\AttributeTok{slope =} \DecValTok{1}\NormalTok{, }\AttributeTok{intercept =} \DecValTok{0}\NormalTok{, }\AttributeTok{linetype =} \StringTok{"dashed"}\NormalTok{, }\AttributeTok{color =} \StringTok{"red"}\NormalTok{) }\SpecialCharTok{+}
  \FunctionTok{labs}\NormalTok{(}\AttributeTok{title =} \StringTok{"Predicted vs. Actual Ecological Footprint"}\NormalTok{,}
       \AttributeTok{x =} \StringTok{"Actual"}\NormalTok{, }\AttributeTok{y =} \StringTok{"Predicted"}\NormalTok{) }\SpecialCharTok{+}
  \FunctionTok{theme\_minimal}\NormalTok{()}
\end{Highlighting}
\end{Shaded}

\begin{verbatim}
## Warning: Removed 3 rows containing missing values or values outside the scale range
## (`geom_point()`).
\end{verbatim}

\includegraphics{Final_Project_files/figure-latex/unnamed-chunk-11-1.pdf}

\begin{Shaded}
\begin{Highlighting}[]
\DocumentationTok{\#\# Calculate performance metrics}
\FunctionTok{library}\NormalTok{(ModelMetrics)}
\end{Highlighting}
\end{Shaded}

\begin{verbatim}
## 
## Attaching package: 'ModelMetrics'
\end{verbatim}

\begin{verbatim}
## The following objects are masked from 'package:caret':
## 
##     confusionMatrix, precision, recall, sensitivity, specificity
\end{verbatim}

\begin{verbatim}
## The following objects are masked from 'package:performance':
## 
##     mae, mse, rmse
\end{verbatim}

\begin{verbatim}
## The following object is masked from 'package:bayestestR':
## 
##     auc
\end{verbatim}

\begin{verbatim}
## The following object is masked from 'package:base':
## 
##     kappa
\end{verbatim}

\begin{Shaded}
\begin{Highlighting}[]
\NormalTok{rmse\_val4 }\OtherTok{\textless{}{-}} \FunctionTok{rmse}\NormalTok{(pol\_test\_clean}\SpecialCharTok{$}\NormalTok{total\_ecological\_footprint, pol\_test\_clean}\SpecialCharTok{$}\NormalTok{pred4) }
\NormalTok{mae\_val4 }\OtherTok{\textless{}{-}} \FunctionTok{mae}\NormalTok{(pol\_test\_clean}\SpecialCharTok{$}\NormalTok{total\_ecological\_footprint, pol\_test\_clean}\SpecialCharTok{$}\NormalTok{pred4)}
\NormalTok{r2\_val4 }\OtherTok{\textless{}{-}} \FunctionTok{cor}\NormalTok{(pol\_test\_clean}\SpecialCharTok{$}\NormalTok{total\_ecological\_footprint, pol\_test\_clean}\SpecialCharTok{$}\NormalTok{pred4)}\SpecialCharTok{\^{}}\DecValTok{2}

\FunctionTok{cat}\NormalTok{(}\StringTok{"Root Mean Square Error (RMSE):"}\NormalTok{, }\FunctionTok{round}\NormalTok{(rmse\_val4, }\DecValTok{2}\NormalTok{), }\StringTok{"}\SpecialCharTok{\textbackslash{}n}\StringTok{"}\NormalTok{)}
\end{Highlighting}
\end{Shaded}

\begin{verbatim}
## Root Mean Square Error (RMSE): 11.02
\end{verbatim}

\begin{Shaded}
\begin{Highlighting}[]
\FunctionTok{cat}\NormalTok{(}\StringTok{"Mean Absolute Error (MAE):"}\NormalTok{, }\FunctionTok{round}\NormalTok{(mae\_val4, }\DecValTok{2}\NormalTok{), }\StringTok{"}\SpecialCharTok{\textbackslash{}n}\StringTok{"}\NormalTok{)}
\end{Highlighting}
\end{Shaded}

\begin{verbatim}
## Mean Absolute Error (MAE): 3.89
\end{verbatim}

\begin{Shaded}
\begin{Highlighting}[]
\FunctionTok{cat}\NormalTok{(}\StringTok{"R{-}squared on Test Set:"}\NormalTok{, }\FunctionTok{round}\NormalTok{(r2\_val4, }\DecValTok{3}\NormalTok{), }\StringTok{"}\SpecialCharTok{\textbackslash{}n}\StringTok{"}\NormalTok{)}
\end{Highlighting}
\end{Shaded}

\begin{verbatim}
## R-squared on Test Set: 0.012
\end{verbatim}

The predictive model performance drops significantly, meaning the model
is likely overfitted to the training data. This overfitting is due to
the complexity of the model, so choosing the next best model, mod3, is
probably a better idea as it has less variable interactions.

Let's try it again with model 3.

\begin{Shaded}
\begin{Highlighting}[]
\NormalTok{train\_mod3 }\OtherTok{\textless{}{-}} \FunctionTok{glm}\NormalTok{(}\AttributeTok{data =}\NormalTok{ pol\_train,}
                  \AttributeTok{formula =}\NormalTok{ mod3}\SpecialCharTok{$}\NormalTok{formula,}
                  \AttributeTok{family =} \StringTok{\textquotesingle{}gaussian\textquotesingle{}}\NormalTok{)}

\NormalTok{pol\_test\_clean}\SpecialCharTok{$}\NormalTok{pred3 }\OtherTok{\textless{}{-}} \FunctionTok{predict}\NormalTok{(train\_mod3, }\AttributeTok{newdata =}\NormalTok{ pol\_test\_clean, }\AttributeTok{type =} \StringTok{"response"}\NormalTok{)}

\FunctionTok{ggplot}\NormalTok{(pol\_test\_clean, }\FunctionTok{aes}\NormalTok{(}\AttributeTok{x =}\NormalTok{ total\_ecological\_footprint, }\AttributeTok{y =}\NormalTok{ pred3)) }\SpecialCharTok{+}
  \FunctionTok{geom\_point}\NormalTok{(}\AttributeTok{color =} \StringTok{"darkblue"}\NormalTok{) }\SpecialCharTok{+}
  \FunctionTok{geom\_abline}\NormalTok{(}\AttributeTok{slope =} \DecValTok{1}\NormalTok{, }\AttributeTok{intercept =} \DecValTok{0}\NormalTok{, }\AttributeTok{linetype =} \StringTok{"dashed"}\NormalTok{, }\AttributeTok{color =} \StringTok{"red"}\NormalTok{) }\SpecialCharTok{+}
  \FunctionTok{labs}\NormalTok{(}\AttributeTok{title =} \StringTok{"Predicted vs. Actual Ecological Footprint"}\NormalTok{,}
       \AttributeTok{x =} \StringTok{"Actual"}\NormalTok{, }\AttributeTok{y =} \StringTok{"Predicted"}\NormalTok{) }\SpecialCharTok{+}
  \FunctionTok{theme\_minimal}\NormalTok{()}
\end{Highlighting}
\end{Shaded}

\includegraphics{Final_Project_files/figure-latex/unnamed-chunk-12-1.pdf}

\begin{Shaded}
\begin{Highlighting}[]
\NormalTok{rmse\_val3 }\OtherTok{\textless{}{-}} \FunctionTok{rmse}\NormalTok{(pol\_test\_clean}\SpecialCharTok{$}\NormalTok{total\_ecological\_footprint, pol\_test\_clean}\SpecialCharTok{$}\NormalTok{pred3) }
\NormalTok{mae\_val3 }\OtherTok{\textless{}{-}} \FunctionTok{mae}\NormalTok{(pol\_test\_clean}\SpecialCharTok{$}\NormalTok{total\_ecological\_footprint, pol\_test\_clean}\SpecialCharTok{$}\NormalTok{pred3)}
\NormalTok{r2\_val3 }\OtherTok{\textless{}{-}} \FunctionTok{cor}\NormalTok{(pol\_test\_clean}\SpecialCharTok{$}\NormalTok{total\_ecological\_footprint, pol\_test\_clean}\SpecialCharTok{$}\NormalTok{pred3)}\SpecialCharTok{\^{}}\DecValTok{2}

\FunctionTok{cat}\NormalTok{(}\StringTok{"Root Mean Square Error (RMSE):"}\NormalTok{, }\FunctionTok{round}\NormalTok{(rmse\_val3, }\DecValTok{2}\NormalTok{), }\StringTok{"}\SpecialCharTok{\textbackslash{}n}\StringTok{"}\NormalTok{)}
\end{Highlighting}
\end{Shaded}

\begin{verbatim}
## Root Mean Square Error (RMSE): 1.36
\end{verbatim}

\begin{Shaded}
\begin{Highlighting}[]
\FunctionTok{cat}\NormalTok{(}\StringTok{"Mean Absolute Error (MAE):"}\NormalTok{, }\FunctionTok{round}\NormalTok{(mae\_val3, }\DecValTok{2}\NormalTok{), }\StringTok{"}\SpecialCharTok{\textbackslash{}n}\StringTok{"}\NormalTok{)}
\end{Highlighting}
\end{Shaded}

\begin{verbatim}
## Mean Absolute Error (MAE): 0.95
\end{verbatim}

\begin{Shaded}
\begin{Highlighting}[]
\FunctionTok{cat}\NormalTok{(}\StringTok{"R{-}squared on Test Set:"}\NormalTok{, }\FunctionTok{round}\NormalTok{(r2\_val3, }\DecValTok{3}\NormalTok{), }\StringTok{"}\SpecialCharTok{\textbackslash{}n}\StringTok{"}\NormalTok{)}
\end{Highlighting}
\end{Shaded}

\begin{verbatim}
## R-squared on Test Set: 0.717
\end{verbatim}

This predictive model is much better, with improved RMSE, MAE,and
R-squared values compared to model 4. This goes to show that modeling
can be extremely precise for given data, but it is not always useful for
predicting hypothetical values.

\end{document}
